\documentclass{article}

\usepackage{hyperref}
\hypersetup{
    colorlinks=true,
    linkcolor=blue,
    filecolor=magenta,      
    urlcolor=cyan,
    pdftitle={How to use renv},
    }

\urlstyle{same}

\title{How to use renv}
\date{2023}
\begin{document}

\maketitle
  
\tableofcontents

\section{Using renv in a Project}
Refer to the \href{https://rstudio.github.io/renv/articles/renv.html#workflow}{Workflow section of the renv website} for more information.

\begin{enumerate}
    \item Install renv (make sure to enter the R REPL by using the \verb|R| command)
    \begin{verbatim}
        R
        install.packages('renv')
    \end{verbatim}
    
    \item Create a new project environment, which is similar to a Python virtual environment
    \begin{verbatim}
        renv::init()
    \end{verbatim}
    
	After initialization, you should see a message to the effect of
	\begin{verbatim}
	    * Project 'PATH/TO/PROJECT' loaded. [renv 0.17.3]
	    * renv activated -- please restart the R session
    \end{verbatim}
    
    \item Install and remove R packages as necessary
    \begin{verbatim}
        R
        install.packages([PACKAGE_NAME])
    \end{verbatim}
    
    \item Save the dependencies/packages used in the project to renv.lock (the lockfile)
    \begin{verbatim}
        renv::snapshot()
    \end{verbatim}
    
    \item Install and remove R packages as necessary
    \item Save the project state by calling renv::snapshot() again, or go back to the previous state as recorded by the lockfile with renv::restore()
\end{enumerate}


\section{How to use renv after Cloning a Project}
\begin{enumerate}
    \item Clone a repository that uses renv
	\begin{verbatim}
	    git clone [REPO_URL]
    \end{verbatim}
    
    \item Restore/install the packages listed in the lockfile
	\begin{verbatim}
        R
        renv::restore()
	\end{verbatim}
\end{enumerate}


\section{Miscellaneous Information}
Most information in this section was pulled from the
\href{https://rstudio.github.io/renv/articles/renv.html#infrastructure}{Infrastructure section of the renv website}.

\begin{itemize}
    \item When committing files to git, make sure to commit the .Rprofile, renv.lock and renv/activate.R files
    \begin{itemize}
        \item Also, it might be helpful to commit the renv/.gitignore file (renv::init will write the required ignore statements to .gitignore)
        \item Note that the renv/library folder is usually ignored
    \end{itemize}
    \item Committing the renv/settings.json file is optional
\end{itemize}

\begin{flushleft}
Refer to \href{https://www.overleaf.com/learn/latex/Learn_LaTeX_in_30_minutes}{this page} for
instructions on how to get started with LaTeX, and \href{https://www.overleaf.com/learn/latex/Hyperlinks}{this page}
for instructions on how to use hyperlinks in a LaTeX document.
\end{flushleft}
\end{document}
