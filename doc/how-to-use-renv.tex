\documentclass{article}
\title{How to use renv}
\date{2023}
\begin{document}
  
\maketitle
  
\tableofcontents

\section{Using renv in a Project}
More information here: \verb|https://rstudio.github.io/renv/articles/renv.html|

\begin{enumerate}
    \item Install renv. make sure to enter the R REPL by using the `R` command.
    \begin{verbatim}
    R
    install.packages('renv')
    \end{verbatim}
    \item Create a new project environment, similar to a Python virtual environment
    \begin{verbatim}renv::init()\end{verbatim}
	
	After initialization, you should see a message to the effect of
	\begin{verbatim}
	* Project 'PATH/TO/PROJECT' loaded. [renv 0.17.3]
	* renv activated -- please restart the R session
    \end{verbatim}
    \item install and remove R packages as necessary
    \item Save the dependencies/packages used in the project to renv.lock (the lockfile)
    \begin{verbatim}renv::snapshot()\end{verbatim}
    \item install and remove R packages as necessary
    \item save the project state by calling renv::snapshot() again, or go back to the previous state as recorded by the lockfile with renv::restore()
\end{enumerate}

\section{How to use renv after Cloning a Project}
\begin{enumerate}
    \item Clone a repository that uses renv
	\begin{verbatim}
	    git clone [URL]
    \end{verbatim}
    \item Restore/install the packages listed in the lockfile
	\begin{verbatim}
    R
    renv::restore()
	\end{verbatim}
\end{enumerate}

\section{Miscellaneous Information}
Most information in this section was pulled from the infrastructure section: \verb|https://rstudio.github.io/renv/articles/renv.html#infrastructure|
\begin{itemize}
    \item When committing files to git, make sure to commit the .Rprofile, renv.lock and renv/activate.R file
    \begin{itemize}
        \item Also, it might be helpful to commit the renv/.gitignore file (renv::init will write the required ignore statements to .gitignore)
        \item Note that the renv/library folder is usually ignored
    \end{itemize}
    \item Committing the renv/settings.json file is optional
\end{itemize}
\end{document}
