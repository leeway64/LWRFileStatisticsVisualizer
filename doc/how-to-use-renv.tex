more information here: https://rstudio.github.io/renv/articles/renv.html
most information in this section was pulled from the infrastructure section: https://rstudio.github.io/renv/articles/renv.html#infrastructure

How to create a new project environment

1. install renv. make sure to enter the R REPL by using the `R` command.
    R
    install.packages('renv')
2. create a new project environment, similar to a Python virtual environment
    renv::init()
	
	you should see a message to the effect of
	* Project 'PATH/TO/PROJECT' loaded. [renv 0.17.3]
	* renv activated -- please restart the R session
3. install and remove R packages as necessary
4. Save the dependencies/packages used in the project to renv.lock (the lockfile)
    renv::snapshot()
5. install and remove R packages as necessary
6. save the project state by calling renv::snapshot() again, or go back to the previous state as recorded
   by the lockfile with renv::restore()

When committing files to git, make sure to commit the .Rprofile, renv.lock and renv/activate.R file.
Also, it might be helpful to commit the renv/.gitignore file. renv::init will write the required ignore statements to .gitignore. the renv/library folder is usually ignored.
committing the renv/settings.json file is optional

How to use renv after cloning

1. clone a repo that uses renv
	git clone ...

2. restore/install the packages listed in the lockfile
	R
	renv::restore()
